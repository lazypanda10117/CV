\documentclass{cv}
\usepackage[left=0.75in,top=0.6in,right=0.75in,bottom=0.6in]{geometry}
\usepackage{amsfonts}
\usepackage{hyperref}
\usepackage{biblatex}
\hypersetup{
    colorlinks=true,
    linkcolor=blue,
    filecolor=magenta,      
    urlcolor=cyan,
}
\urlstyle{same}

\name{Jeffrey Kam}
\addbibresource{bib.bib}

\begin{document}
% empty line needed

% Waterloo, Ontario \\
% Canada \\
\href{hykam@uwaterloo.ca}{hykam@uwaterloo.ca} \\
\href{https://jeffreyhykam.com}{https://jeffreyhykam.com}

\begin{rSection}{Interests}
	I am mainly interested in graph theory and its algorithmic implications in areas such as 
	algorithm design, coding theory, and quantum computing. I am also keen on various topics in computer algebra.
\end{rSection}

\begin{rSection}{Education}
\begin{rSubsection}{University of Waterloo}{Sep 2017 - Present}{Currently in fourth year}{}
	\item Double major in Combinatorics \& Optimization and Computer Science
	\item Minor in Pure Mathematics
	\item Term Dean's Honours List
\end{rSubsection}

\begin{rSubsection}{Relevant Courses}{}{}{}
	\item Graduate: graph-theoretic algorithms, algorithms for graph minors
	\item Undergraduate: coding theory, fields and galois theory, algebraic graph theory, network flow theory, 
	algebraic number theory, introduction to graph theory, neural networks, algorithms
\end{rSubsection}
\end{rSection}

\begin{rSection}{Research Experience}
\begin{rSubsection}{University of Waterloo - Softwre REBELs}{May 2021 - present}{Undergraduate Research Fellow \\ Supervised by Prof. Shane McIntosh}{Waterloo, Canada}
	\item Designing a new architecture for extracting preprocessor definitions and augmenting CMAKE dependency graph
	using Python, ANTLR, and Neo4j.
\end{rSubsection}

\begin{rSubsection}{University of Waterloo - Symbolic Computation Group}{May 2020 - Apr 2021}{Undergraduate Research Assisstant (Part-time)\\ Supervised by Dr. Armin Jamshidpey}{Waterloo, Canada}
    \item Understanding various methods on testing normal elements in $\mathbb{F}_{p^n}$ and its connections to circulant matrices and primitive elements.
  	\item Researching different algorithms to find Smith Normal Form over $\mathbb{Z}_{p^2}$ efficiently, such as experimenting with $J$-ideal.
\end{rSubsection}

\begin{rSubsection}{BlackBerry - Security Research Group}{Janurary 2020 - April 2020}{Security Researcher Intern \\ Supervised by Shay Berkovich and Dr. Glenn Wurster}{Waterloo, Canada}
	\item Researched and designed a universal benchmark to quantitatively measure the effectiveness and accuracy of container image scanners.
	\item Analyzed techniques of image inspection and vulnerability scanning through open source technologies.
	% \item Designed a universal import framework for Anchore Engine to extend our scanning capabilities
	\item Researched on utilizing machine learning for fuzzing algorithmic complexity vulnerabilities.
	% \item Presented to the security research group on current developments of ML-based fuzzing and fuzzing techniques for ACVs, along with potential problems, experiments, and optimizations
\end{rSubsection}
\end{rSection}

\begin{rSection}{Publications}
\begin{rSubsectionPure}
	\item \textbf{{UBCIS}: Ultimate Benchmark for Container Image Scanning}, \\
	with Shay Berkovich and Glenn Wurster \\
	Published in 13th {USENIX} Workshop on Cyber Security Experimentation and Test ({CSET} 20). 
	% \href{https://www.usenix.org/conference/cset20/presentation/berkovich}{https://www.usenix.org/conference/cset20/presentation/berkovich}
\end{rSubsectionPure}

\begin{rSubsectionPure}
	\item \textbf{bioSyntax: Syntax Highlighting For Computational Biology}, \\
	with A. Babaian, et al. \\
	Published in BMC Bioinformatics 19, 303 (2018). 
	% \href{https://doi.org/10.1186/s12859-018-2315-y}{https://doi.org/10.1186/s12859-018-2315-y}
\end{rSubsectionPure}
\end{rSection}

\begin{rSection}{Relevant Projects and Presentations}
\begin{rSubsection}{A graph-theoretic proof for an upper bound of maximum block code size}{(4 pages)}{}{}
	\item An alternative graph-theoretic proof to the linear-algebraic proof for the upper bound of $T_q(n,d)$ under 
	the restriction $d > \frac{n(q-1)}{q}$ using Tur\'{a}n's theorem.
\end{rSubsection}
	
\begin{rSubsection}{Network coding, network flow, and matroid theory}{(37 pages)}{}{}
	\item A survey on the link between the study of network coding, network flow, and matroid theory 
	to better understand the limits of network coding. It covers network coding fundamentals, network flow in 
	multicast networks, matroidal network, and some results on the computational complexity and network capacity.
\end{rSubsection}

\begin{rSubsection}{Bounding queue-number in planar graphs}{(23 pages)}{}{}
	\item A written report of a recent proof by Dujmovi\'{c} et al. for a 20-year old conjecture on the queue-number of planar graphs, 
	accompanied by lecture videos.
\end{rSubsection}

\begin{rSubsection}{Deciding tangles with weighted vertex sets and certifying large branchwidth}{(14 pages)}{}{}
	\item A report on Elbracht et al.'s partial solution to finding a vertex subset characterization of a tangle, and Oum and 
	Seymour's paper on certifying large branch-width in polynomial time with tangle-kits.
\end{rSubsection}

\begin{rSubsection}{Eigenvalues and Graph Bisection}{(13 pages)}{}{}
	\item A presentation on the proof of $2$ claims from Boppana's paper "Eigenvalues and Graph Bisection: An Average-case Analysis", 
	where the details are omitted by the author.
\end{rSubsection}
\end{rSection}


\begin{rSection}{Awards and Distinctions}
\begin{rSubsection}{University of Waterloo}{May 2021}{Undergraduate Research Fellowship}{}
	\item Based on academic performance and research potentials.
\end{rSubsection}

\begin{rSubsection}{University of Waterloo}{Dec 2020}{Frank Lun Scholarship for Excellence}{}
	\item Based on academic performance and demonstrated leadership abilities
\end{rSubsection}

% \begin{rSubsection}{University of Waterloo}{Sep 2017}{University of Waterloo President's Scholarship}{}
% 	\item Based on entrance average
% \end{rSubsection}

% \begin{rSubsection}{HackSeq}{Sep 2017}{Winning Team in HackSeq 2017}{}
% 	\item Bioinformatics competiton in UBC
% \end{rSubsection}

\begin{rSubsection}{University of Hong Kong and University of Waterloo}{Mar 2017}{Honourable Mention in Canadian Computing Competition Hong Kong}{}
	\item Based on performance in the Canadian Computing Competition 
\end{rSubsection}
\end{rSection}


\begin{rSection}{Professional Experience}
\begin{rSubsection}{GTS}{Sep 2020 - Dec 2020}{Software Engineering Intern}{New York, US}
	\item Worked on the core trading engine using C++ and Python.
\end{rSubsection}

\begin{rSubsection}{Zenefits}{May 2019 - Aug 2019}{Software Engineering Intern}{Vancouver, Canada}
	\item Developed new permission guards in Python involving distributed message queue systems.
	\item Designed new customer-facing features with React and integrated with the backend through GraphQL.
\end{rSubsection}

\begin{rSubsection}{Horizn}{May 2018 - Aug 2018}{Software Developer Intern}{Toronto, Canada}
	\item Wrote automation scripts and queries to streamline clients’ data transfer to AWS services. 
\end{rSubsection}
\end{rSection}

\begin{rSection}{Technical Skills}
\begin{tabular}{ @{} >{\bfseries}l @{\hspace{6ex}} l }
	Programming & Python, C++ (Boost), SAGE, Racket, \LaTeX \\
	Tools & Git, C++ tools (GCC, GDB), Docker, Linux, PLY, ANTLR
\end{tabular}
\end{rSection}

\newpage
\printbibliography

\end{document}
