\documentclass{cv}
\usepackage[left=0.75in,top=0.6in,right=0.75in,bottom=0.6in]{geometry}
\usepackage{biblatex}
\usepackage{hyperref}
\hypersetup{
    colorlinks=true,
    linkcolor=blue,
    filecolor=magenta,      
    urlcolor=cyan,
}
\urlstyle{same}

\name{Jeffrey Kam}
\address{\href{mailto:hykam@uwaterloo.ca}{hykam@uwaterloo.ca} \\
UWaterloo ID: 20716434}
\begin{document}

\begin{rSection}{Interests}
I am interested in graph theory and graph algorithms research, with a particular interest in structural graph theory, graph drawing, and graph colouring. In addition, I am also keen on quantum computing and hope to learn more through courses like quantum information processing and algebraic graph theory in the future.
\end{rSection}

\begin{rSection}{Education}
\begin{rSubsection}{University of Waterloo}{Sep 2017 - Present}{Currently in term 4A}{}
	\item Candidate for B.Math. in Combinatorics \& Optimizations and Computer Science
	\item Minor in Pure Mathematics
	% \item MAV: 87 \% and 83 \%
	% \item Term Dean's Honours List
\end{rSubsection}

\begin{rSubsection}{Relevant Courses}{}{Taken already / Expected before Summer 2021.
}{}
	\item Graph-theoretic Algorithms - CS762
	\item Algorithms for Graph Minors - CO749
	\item Algebraic Graph Theory - CO444 (expected)
	\item Integer Programming - CO452 (expected)
	\item Introduction to Graph Theory - CO342
	\item Network Flow Theory - CO351
	\item Algorithms - CS341
	\item Statistical and Computational Foundations of Machine Learning - CS485
	\item Groups and Rings - PMATH347
	\item Algebraic Number Theory - PMATH441
	\item Complex Analysis - PMATH352 (expected)
	\item Formal Languages - CS462 (expected)
\end{rSubsection}

\begin{rSubsection}{Relevant Projects}{}{}{}
	\item \textbf{Bounded Queue-number in Planar Graphs (CS762)} - \\
	Explore a recent proof by Dujmovi\'{c} et al for a 20-year old conjectjure on whether the queue-number of planar graph is bounded. 
	\item \textbf{Tangles in Graph Minor X (CO749)} - \\
	Explore a new notion of connectivity in graph that arises in the Graph Minor project by Robertson and Seymour. 
\end{rSubsection}
\end{rSection}

\begin{rSection}{Publication}
\begin{rSubsectionPure}
	\item \textbf{{UBCIS}: Ultimate Benchmark for Container Image Scanning}, \\
	with Shay Berkovich and Glenn Wurster \\
	Published in 13th {USENIX} Workshop on Cyber Security Experimentation and Test ({CSET} 20). \\
	\href{https://www.usenix.org/conference/cset20/presentation/berkovich}{https://www.usenix.org/conference/cset20/presentation/berkovich}
\end{rSubsectionPure}

\begin{rSubsectionPure}
	\item \textbf{bioSyntax: Syntax Highlighting For Computational Biology}, \\
	with A. Babaian, et al. \\
	Published in BMC Bioinformatics 19, 303 (2018). \\
	\href{https://doi.org/10.1186/s12859-018-2315-y}{https://doi.org/10.1186/s12859-018-2315-y}
\end{rSubsectionPure}
\end{rSection}

\begin{rSection}{Research Experience}
\begin{rSubsection}{University of Waterloo - Symbolic Computation Group}{May 2020 - Sep 2020}{Undergraduate Research Assistant}{Waterloo, Canada}
	\item Experiment with $J$-ideal and Smith Normal Form using SAGE.
  	\item Understand relationships between matrix normal forms and ideals.
\end{rSubsection}

\begin{rSubsection}{BlackBerry - Security Research Group}{Janurary 2020 - April 2020}{Security Research Intern}{Waterloo, Canada}
	\item Researched and designed a universal benchmark to quantitatively measure the effectiveness and accuracy of container image scanners
	\item Analyzed techniques of image inspection and vulnerability scanning through open source technologies
	\item Designed a universal import framework for Anchore Engine to extend our scanning capabilities
	\item Researched on utilizing machine learning for fuzzing algorithmic complexity vulnerabilities (ACV)
	\item Presented to the security research group on current developments of machine-learning-based fuzzing and fuzzing techniques for ACVs, along with potential problems, experiments, and optimizations.
\end{rSubsection}
\end{rSection}

\begin{rSection}{Work Experience}
\begin{rSubsection}{GTS}{Sep 2020 - Present}{Software Engineering Intern}{Remote (New York, US)}
	\item Working on performant C++ and Python code for the core trading engine
\end{rSubsection}

\begin{rSubsection}{Zenefits}{May 2019 - Aug 2019}{Software Engineering Intern}{Vancouver, Canada}
	\item Developed new permission services in Django with extensive unit tests to guard against unauthorized edits of review data
	% \item Developed authentication and mailing services for Kudos, a social network prototype for providing recognition
	% \item Built features for Zenefits \textit{Performance Management} with React and integrate with the backend through GraphQL services
	\item Designed a sequeitial document update service using a distributed messsage queue system Celery
\end{rSubsection}

\begin{rSubsection}{Horizn}{May 2018 - Aug 2018}{Web Developer Intern}{Toronto, Canada}
	\item Built Laravel components for internal app and wrote Python scripts to transfer clients’ data in AWS
	\item Wrote automation scripts to scrape data from files and database and compile them into json files
	% \item Helped built new features/products to clients’ LAMP-based websites
\end{rSubsection}
\end{rSection}

\begin{rSection}{Awards}
\begin{rSubsectionPure}
	\item First place in HackSeq 2017 bioinformatics competiton in UBC
	\item Honourable mention in Canadian Computing Competition Hong Kong 2017
	\item University of Waterloo President's Scholarship
\end{rSubsectionPure}
\end{rSection}

% \begin{rSection}{Projects}
% \begin{rSubsection}{Statistical Analysis on Amazon Marketplace Data}{}{}{}
% 	\item Analyzed Amazon marketplace data with using Python frameworks, such as Numpy and Pandas 
% 	\item Employed various mathematical methods, such as PCA, time series, decision tree, and sentiment analysis
% \end{rSubsection}
% \begin{rSubsection}{Scheduler}{}{}{}{}
% 	\item Researched on methods for building and optimizing scheduler for block-based class schedule
% 	\item Wrote a class scheduler in JAVA using Genetic Algorithm derived from the book \textit{Genetic Algorithm for Java Basics}
% \end{rSubsection}
% \begin{rSubsection}{Subset C Compiler}{}{}{}
% 	\item Wrote a Subset C compiler using functional programming language Racket
% 	\item Optimized compiler performance for functional-style code, using dead code detection, expression reduction, and more.
% \end{rSubsection}
% \end{rSection}

\begin{rSection}{Skills}
\begin{tabular}{ @{} >{\bfseries}l @{\hspace{6ex}} l }
	Programming & Python, C++, SAGE, Scheme \\
	Tools & Git, C++ tools (i.e. GCC, GDB, Valgrind), Docker, Linux, Jupyter
\end{tabular}
\end{rSection}

\end{document}
