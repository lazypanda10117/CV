\documentclass{cv}
\usepackage[left=0.75in,top=0.6in,right=0.75in,bottom=0.6in]{geometry}
\usepackage{amsfonts}
\usepackage{hyperref}
\usepackage{biblatex}
\hypersetup{
    colorlinks=true,
    linkcolor=blue,
    filecolor=magenta,      
    urlcolor=cyan,
}
\urlstyle{same}

\name{Jeffrey Kam}
\address{\href{mailto:hykam@uwaterloo.ca}{hykam@uwaterloo.ca}}

\addbibresource{bib.bib}

\begin{document}

\begin{rSection}{Interests}
	I am interested in graph theory and would like to work on topics related to graph algorithms, graph colouring, and structural graph theory. In addition, I am also keen on topics related to discrete optimization.
\end{rSection}

\begin{rSection}{Education}
% UWaterloo version
\begin{rSubsection}{University of Waterloo}{Sep 2017 - Present}{Currently in term 4A}{}
	\item Candidate for BMath. Combinatorics \& Optimization and Computer Science
	\item Minor in Pure Mathematics
	\item Term Dean's Honours List
\end{rSubsection}

% External
% \begin{rSubsection}{University of Waterloo}{Sep 2017 - Present}{Currently in 4^{th} year}{}
% 	\item Double major in Combinatorics \& Optimization and Computer Science
% 	\item Minor in Pure Mathematics
% 	\item Term Dean's Honours List
% \end{rSubsection}

% UWaterloo version
\begin{rSubsection}{Relevant Courses}{}{Taken already / Expected before Summer 2021.}{E = Expected}
	\item Graph-theoretic Algorithms - CS762
	\item Algorithms for Graph Minors - CO749
	\item Algebraic Graph Theory - CO444 [E]
	\item Network Flow Theory - CO351
	\item Introduction to Graph Theory - CO342
	\item Coding Theory - CO331 [E]
	\item Introduction to Optimization - CO250
\end{rSubsection}

% External
% \begin{rSubsection}{Relevant Courses}{}{Taken already / Expected before Summer 2021.}{G = Graduate}
% 	\item Graph-theoretic Algorithms - CS762 [G]
% 	\item Algorithms for Graph Minors - CO749 [G]
% 	\item Algebraic Graph Theory - CO444
% 	\item Network Flow Theory - CO351
% 	\item Introduction to Graph Theory - CO342
% 	\item Coding Theory - CO331
% 	\item Introduction to Optimization - CO250
% \end{rSubsection}

\begin{rSubsection}{Relevant Projects}{}{}{}	
	\item \textbf{Bounded queue-number in planar graphs (CS762)} - \href{https://jeffreyhykam.com/writings/}{Project Page}\\
	I explored a recent proof by Dujmovi\'{c} et al \cite{queue} for a 20-year old conjectjure on whether the queue-number of planar graphs is bounded. I also wrote lecture notes and made a lecture video on the topic.\\

	\item \textbf{Deciding tangles with weighted vertex sets (CO749)} - \href{https://jeffreyhykam.com/writings/}{Project Page}\\
	I wrote a report on Elbracht et al.'s fractional solution \cite{tangle} to an open problem about finding a vertex subset $X$ that characterizes a tangle by checking which side of a separation has more vertices in $X$. In addition, I also explored Oum and Seymour's paper \cite{branchwidth} on certifying large branch-width in polynomial time with tangle-kits. 
\end{rSubsection}
\end{rSection}

\begin{rSection}{Publications}
\begin{rSubsectionPure}
	\item \textbf{{UBCIS}: Ultimate Benchmark for Container Image Scanning}, \\
	with Shay Berkovich and Glenn Wurster \\
	Published in 13th {USENIX} Workshop on Cyber Security Experimentation and Test ({CSET} 20). \\
	\href{https://www.usenix.org/conference/cset20/presentation/berkovich}{https://www.usenix.org/conference/cset20/presentation/berkovich}
\end{rSubsectionPure}

\begin{rSubsectionPure}
	\item \textbf{bioSyntax: Syntax Highlighting For Computational Biology}, \\
	with A. Babaian, et al. \\
	Published in BMC Bioinformatics 19, 303 (2018). \\
	\href{https://doi.org/10.1186/s12859-018-2315-y}{https://doi.org/10.1186/s12859-018-2315-y}
\end{rSubsectionPure}
\end{rSection}

\begin{rSection}{Research Experience}
\begin{rSubsection}{University of Waterloo - Symbolic Computation Group}{May 2020 - Sep 2020}{Undergraduate Research Assisstant \\ Supervised by Armin Jamshidpey}{Waterloo, Canada}
  	\item Researched different methods to find Smith Normal Form over $\mathbb{Z}_{p^2}$ efficiently, such as experimenting with probabilistic algorithms and utilizing $J$-ideal
  	\item Investigate new efficient methods of finding normal bases in $\mathbb{F}_{p^n}$ and revisited various topics in abstract algebra and Galois theory
\end{rSubsection}

\begin{rSubsection}{BlackBerry - Security Research Group}{Janurary 2020 - April 2020}{Security Research Intern \\ Supervised by Shay Berkovich and Glenn Wurster}{Waterloo, Canada}
	\item Researched and designed a universal benchmark to quantitatively measure the effectiveness and accuracy of container image scanners
	\item Analyzed techniques of image inspection and vulnerability scanning through open source technologies
	\item Designed a universal import framework for Anchore Engine to extend our scanning capabilities
	\item Researched on utilizing machine learning for fuzzing algorithmic complexity vulnerabilities (ACV) by reading multiple security-related journals and conference papers
	\item Presented to the security research group on current developments of ML-based fuzzing and fuzzing techniques for ACVs, along with potential problems, experiments, and optimizations
\end{rSubsection}
\end{rSection}

\begin{rSection}{Work Experience}
\begin{rSubsection}{GTS}{Sep 2020 - Present}{Software Engineering Intern}{New York, US}
	\item Working on performant C++ and Python code for the core trading engine
\end{rSubsection}

\begin{rSubsection}{Zenefits}{May 2019 - Aug 2019}{Software Engineering Intern}{Vancouver, Canada}
	\item Developed new permission services in Python to guard against unauthorized review editing
	\item Designed a sequeitial document update service using a distributed messsage queue system
\end{rSubsection}

\begin{rSubsection}{Horizn}{May 2018 - Aug 2018}{Software Developer Intern}{Toronto, Canada}
	\item Wrote automation scripts in Python to scrape data from files and database into JSON files
	\item Learned foundational object-oriented programming concepts, such as factory and observer pattern
\end{rSubsection}
\end{rSection}

\begin{rSection}{Awards}
\begin{rSubsectionPure}
	\item First place in HackSeq 2017 bioinformatics competiton in UBC
	\item Honourable mention in Canadian Computing Competition Hong Kong 2017
\end{rSubsectionPure}
\end{rSection}

\begin{rSection}{Technical Skills}
\begin{tabular}{ @{} >{\bfseries}l @{\hspace{6ex}} l }
	Programming & Python, C++, SAGE, Scheme \\
	Tools & Git, C++ tools (i.e. GCC, GDB, Valgrind), Docker, Linux, Jupyter
\end{tabular}
\end{rSection}

\newpage
\printbibliography

\end{document}
