\documentclass{cv}
\usepackage[left=0.75in,top=0.6in,right=0.75in,bottom=0.6in]{geometry}
\usepackage{amsfonts}
\usepackage{hyperref}
\usepackage{biblatex}
\hypersetup{
    colorlinks=true,
    linkcolor=blue,
    filecolor=magenta,      
    urlcolor=cyan,
}
\urlstyle{same}

\name{Jeffrey Kam}
\addbibresource{bib.bib}

\begin{document}
% empty line needed

Waterloo, Ontario \\
Canada \\
\href{hykam@uwaterloo.ca}{hykam@uwaterloo.ca}

\begin{rSection}{Interests}
	I am interested in the intersection of mathematics and computer science. In particular, this includes the study of graph structures, graph algorithms, and discrete optimization.
\end{rSection}

\begin{rSection}{Education}
\begin{rSubsection}{University of Waterloo}{Sep 2017 - Present}{Currently in fourth year}{}
	\item Double Major in Combinatorics \& Optimization and Computer Science
	\item Minor in Pure Mathematics
	\item Term Dean's Honours List
\end{rSubsection}

\begin{rSubsection}{Relevant Courses}{}{}{}
	\item Graduate: graph-theoretic algorithms, algorithms for graph minors
	\item Undergraduate: algebraic graph theory, network flow theory, coding theory, introduction to graph theory, neural networks, algorithms
\end{rSubsection}

\begin{rSubsection}{Relevant Projects}{}{}{}	
	\item \textbf{Bounded queue-number in planar graphs}\\
	I explored a recent proof by Dujmovi\'{c} et al \cite{queue} for a 20-year old conjectjure on whether the queue-number of planar graphs is bounded. I also wrote lecture notes and made a lecture video on the topic.\\

	\item \textbf{Deciding tangles with weighted vertex sets}\\
	I wrote a report on Elbracht et al.'s fractional solution \cite{tangle} to an open problem about finding a vertex subset that characterizes a tangle by a majority vote. In addition, I also explored Oum and Seymour's paper \cite{branchwidth} on certifying large branch-width in polynomial time with tangle-kits. 
\end{rSubsection}
\end{rSection}

\begin{rSection}{Publications}
\begin{rSubsectionPure}
	\item \textbf{{UBCIS}: Ultimate Benchmark for Container Image Scanning}, \\
	with Shay Berkovich and Glenn Wurster \\
	Published in 13th {USENIX} Workshop on Cyber Security Experimentation and Test ({CSET} 20). \\
	\href{https://www.usenix.org/conference/cset20/presentation/berkovich}{https://www.usenix.org/conference/cset20/presentation/berkovich}
\end{rSubsectionPure}

\begin{rSubsectionPure}
	\item \textbf{bioSyntax: Syntax Highlighting For Computational Biology}, \\
	with A. Babaian, et al. \\
	Published in BMC Bioinformatics 19, 303 (2018). \\
	\href{https://doi.org/10.1186/s12859-018-2315-y}{https://doi.org/10.1186/s12859-018-2315-y}
\end{rSubsectionPure}
\end{rSection}

\begin{rSection}{Research Experience}
\begin{rSubsection}{University of Waterloo}{May 2021 - present}{Undergraduate Research Fellow \\ Supervised by Shane McIntosh}{Waterloo, Canada}
	\item To be filled
\end{rSubsection}

\begin{rSubsection}{University of Waterloo - Symbolic Computation Group}{May 2020 - present}{Undergraduate Research Assisstant (Part-time)\\ Supervised by Armin Jamshidpey}{Waterloo, Canada}
  	\item Investigate new efficient methods of finding normal bases in $\mathbb{F}_{p^n}$ and revisited various topics in abstract algebra and Galois theory\item Researched different methods to find Smith Normal Form over $\mathbb{Z}_{p^2}$ efficiently, such as experimenting with probabilistic algorithms and utilizing $J$-ideal
\end{rSubsection}

\begin{rSubsection}{BlackBerry - Security Research Group}{Janurary 2020 - April 2020}{Security Researcher Intern \\ Supervised by Shay Berkovich and Glenn Wurster}{Waterloo, Canada}
	\item Researched and designed a universal benchmark to quantitatively measure the effectiveness and accuracy of container image scanners
	\item Analyzed techniques of image inspection and vulnerability scanning through open source technologies
	% \item Designed a universal import framework for Anchore Engine to extend our scanning capabilities
	\item Researched on utilizing machine learning for fuzzing algorithmic complexity vulnerabilities (ACV) by reading multiple security-related journals and conference papers
	\item Presented to the security research group on current developments of ML-based fuzzing and fuzzing techniques for ACVs, along with potential problems, experiments, and optimizations
\end{rSubsection}
\end{rSection}

\begin{rSection}{Professional Experience}
\begin{rSubsection}{GTS}{Sep 2020 - Dec 2020}{Software Engineering Intern}{New York, US}
	\item Worked on high-performance C++ and Python code for the core trading engine. (details undisclosed)
	% involving code generation, AST construction, and grammar design.
\end{rSubsection}

\begin{rSubsection}{Zenefits}{May 2019 - Aug 2019}{Software Engineering Intern}{Vancouver, Canada}
	\item Developed new permission services in Python to guard against unauthorized review editing
	\item Designed a sequeitial document update service using a distributed messsage queue system
\end{rSubsection}

\begin{rSubsection}{Horizn}{May 2018 - Aug 2018}{Software Developer Intern}{Toronto, Canada}
	\item Wrote automation scripts in Python to scrape data from files and database into JSON files
	\item Learned foundational object-oriented programming concepts, such as factory and observer pattern
\end{rSubsection}
\end{rSection}

\begin{rSection}{Awards and Scholarships}
\begin{rSubsection}{University of Waterloo}{May 2021}{Undergraduate Research Fellowship}{}
	\item Awarded for research excellence
\end{rSubsection}

\begin{rSubsection}{University of Waterloo}{Dec 2020}{Frank Lun Scholarship for Excellence}{}
	\item Awarded based on academic performance and leadership abilities
\end{rSubsection}

% \begin{rSubsection}{HackSeq}{Sep 2017}{Winning Team in HackSeq 2017}{}
% 	\item Bioinformatics competiton in UBC
% \end{rSubsection}

\begin{rSubsection}{University of Hong Kong and University of Waterloo}{Mar 2017}{Honourable Mention in Canadian Computing Competition Hong Kong}{}
	\item Awarded based on performance in the programming contest 
\end{rSubsection}

\end{rSection}

\begin{rSection}{Technical Skills}
\begin{tabular}{ @{} >{\bfseries}l @{\hspace{6ex}} l }
	Programming & Python, C++ (Boost), SAGE, Scheme, \LaTeX \\
	Tools & Git, C++ tools (i.e. GCC, GDB), Docker, Linux, PLY Jupyter
\end{tabular}
\end{rSection}

\newpage
\printbibliography

\end{document}
