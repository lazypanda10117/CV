\documentclass{cv}
\usepackage[left=0.75in,top=0.6in,right=0.75in,bottom=0.6in]{geometry}
\usepackage{amsfonts}
\usepackage{hyperref}
\usepackage{biblatex}
\hypersetup{
    colorlinks=true,
    linkcolor=blue,
    filecolor=magenta,      
    urlcolor=cyan,
}
\urlstyle{same}

\name{Jeffrey Kam}
\addbibresource{bib.bib}

\begin{document}
% empty line needed

% Waterloo, Ontario \\
% Canada \\
\href{hykam@uwaterloo.ca}{hykam@uwaterloo.ca} \\
\href{https://jeffreyhykam.com}{https://jeffreyhykam.com}

\begin{rSection}{Interests}
	I am mainly interested in graph theory and its algorithmic implications, such as algorithm design, coding theory, and discrete optimization. 
	I am also keen on various topics in computer algebra.
\end{rSection}

\begin{rSection}{Education}
\begin{rSubsection}{University of Waterloo}{Sep 2017 - Present}{Currently in fourth year}{}
	\item Double major in Combinatorics \& Optimization and Computer Science
	\item Minor in Pure Mathematics
	\item Dean's Honours List
\end{rSubsection}

\begin{rSubsection}{Relevant Courses}{}{}{}
	\item Graduate: graph-theoretic algorithms, algorithms for graph minors
	\item Undergraduate: algebraic graph theory, matroid theory, network flow theory, coding theory, algebraic number theory, neural networks, statistical foundation for machine learning, algorithms
\end{rSubsection}

\begin{rSubsection}{Relevant Projects}{}{}{}
	\item \textbf{CS762: Graph-theoretic Algorithms} \\
	"Bounding queue-number in planar graphs" \\
	An exploration of a recent proof by Dujmovi\'{c} et al. for a 20-year old conjectjure on the queue-number of planar graphs.
	\item \textbf{CO749: Algorithms for graph minors} \\
	"Deciding tangles with weighted vertex sets and certifying large branchwidth with tangle-kits" \\
	A report on Elbracht et al.'s partial solution to finding a vertex subset characterization of a tangle, 
	and Oum and Seymour's paper on certifying large branch-width in polynomial time with tangle-kits.
	\item \textbf{CO331: Coding Theory} \\
	"Connections between network coding and matroid theory" \\
	An investigation on the link between the study of network coding and matroid theory 
	to better understand the limits of network coding.
\end{rSubsection}
\end{rSection}

\begin{rSection}{Research Experience}
\begin{rSubsection}{University of Waterloo}{May 2021 - present}{Undergraduate Research Fellow \\ Supervised by Prof. Shane McIntosh}{Waterloo, Canada}
	\item Analyzing features within a code graph. (Details to be included)
\end{rSubsection}

\begin{rSubsection}{University of Waterloo - Symbolic Computation Group}{May 2020 - present}{Undergraduate Research Assisstant (Part-time)\\ Supervised by Dr. Armin Jamshidpey}{Waterloo, Canada}
    \item Investigate new efficient methods of finding normal bases in $\mathbb{F}_{p^n}$ and connections between circulant matrices and primitive elements.
  	\item Researched different methods to find Smith Normal Form over $\mathbb{Z}_{p^2}$ efficiently, such as experimenting with $J$-ideal.
\end{rSubsection}

\begin{rSubsection}{BlackBerry - Security Research Group}{Janurary 2020 - April 2020}{Security Researcher Intern \\ Supervised by Shay Berkovich and Dr. Glenn Wurster}{Waterloo, Canada}
	\item Researched and designed a universal benchmark to quantitatively measure the effectiveness and accuracy of container image scanners.
	\item Analyzed techniques of image inspection and vulnerability scanning through open source technologies.
	% \item Designed a universal import framework for Anchore Engine to extend our scanning capabilities
	\item Researched on utilizing machine learning for fuzzing algorithmic complexity vulnerabilities.
	% \item Presented to the security research group on current developments of ML-based fuzzing and fuzzing techniques for ACVs, along with potential problems, experiments, and optimizations
\end{rSubsection}
\end{rSection}

\begin{rSection}{Publications}
\begin{rSubsectionPure}
	\item \textbf{{UBCIS}: Ultimate Benchmark for Container Image Scanning}, \\
	with Shay Berkovich and Glenn Wurster \\
	Published in 13th {USENIX} Workshop on Cyber Security Experimentation and Test ({CSET} 20). 
	% \href{https://www.usenix.org/conference/cset20/presentation/berkovich}{https://www.usenix.org/conference/cset20/presentation/berkovich}
\end{rSubsectionPure}

\begin{rSubsectionPure}
	\item \textbf{bioSyntax: Syntax Highlighting For Computational Biology}, \\
	with A. Babaian, et al. \\
	Published in BMC Bioinformatics 19, 303 (2018). 
	% \href{https://doi.org/10.1186/s12859-018-2315-y}{https://doi.org/10.1186/s12859-018-2315-y}
\end{rSubsectionPure}
\end{rSection}

% \begin{rSection}{Relevant Projects}
% \begin{rSubsection}{Bounding queue-number in planar graphs}{}{}{}
% 	\item An exploration of a recent proof by Dujmovi\'{c} et al. for a 20-year old conjectjure on whether the queue-number of planar graphs is bounded, accompined by lecture notes and videos.
% \end{rSubsection}

% \begin{rSubsection}{Deciding tangles with weighted vertex sets}{}{}{}
% 	\item A report on Elbracht et al.'s partial solution to finding a vertex subset characterization of a tangle, and Oum and Seymour's paper on certifying large branch-width in polynomial time with tangle-kits.
% \end{rSubsection}
% \end{rSection}

\begin{rSection}{Awards and Distinctions}
\begin{rSubsection}{University of Waterloo}{May 2021}{Undergraduate Research Fellowship}{}
	\item Based on academic performance and research abilities
\end{rSubsection}

\begin{rSubsection}{University of Waterloo}{Dec 2020}{Frank Lun Scholarship for Excellence}{}
	\item Based on academic performance and demonstrated leadership abilities
\end{rSubsection}

% \begin{rSubsection}{University of Waterloo}{Sep 2017}{University of Waterloo President's Scholarship}{}
% 	\item Based on entrance average
% \end{rSubsection}

% \begin{rSubsection}{HackSeq}{Sep 2017}{Winning Team in HackSeq 2017}{}
% 	\item Bioinformatics competiton in UBC
% \end{rSubsection}

\begin{rSubsection}{University of Hong Kong and University of Waterloo}{Mar 2017}{Honourable Mention in Canadian Computing Competition Hong Kong}{}
	\item Based on performance in the Canadian Computing Competition 
\end{rSubsection}
\end{rSection}


\begin{rSection}{Professional Experience}
\begin{rSubsection}{GTS}{Sep 2020 - Dec 2020}{Software Engineering Intern}{New York, US}
	\item Worked on the core trading engine involving code generation. (details undisclosed)
	% involving code generation, AST construction, and grammar design.
\end{rSubsection}

\begin{rSubsection}{Zenefits}{May 2019 - Aug 2019}{Software Engineering Intern}{Vancouver, Canada}
	\item Developed new permission guards in Python involving distributed messsage queue systems
\end{rSubsection}

\begin{rSubsection}{Horizn}{May 2018 - Aug 2018}{Software Developer Intern}{Toronto, Canada}
	\item Wrote automation scripts and queries to streamline clients’ data transfer to AWS services 
\end{rSubsection}
\end{rSection}

\begin{rSection}{Technical Skills}
\begin{tabular}{ @{} >{\bfseries}l @{\hspace{6ex}} l }
	Programming & Python, C++ (Boost), SAGE, Scheme, \LaTeX \\
	Tools & Git, C++ tools (GCC, GDB), Docker, Linux, PLY, ANTLR
\end{tabular}
\end{rSection}

\newpage
\printbibliography

\end{document}
